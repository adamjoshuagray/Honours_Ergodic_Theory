\documentclass{unswmaths}
\usepackage{unswshortcuts}
\begin{document}
\author{Adam J. Gray}
\title{Assignment 1}
\subject{Ergodic Theory}
\studentno{3329798}

\unswtitle

\section{}
Let $ (X_1, \mathcal{B}_1, \mu_1, T_1) $ and $(X_2, \mathcal{B}_2, \mu_1, T_2)$ be two measure theoretically isomorphic p.p.ts ( $ \pi : X_1 \lra X_2 $ ). Suppose $ (X_1, \mathcal{B}_1, \mu_1, T_1) $ is mixing.

Then for $ E, F \in \mathcal{B}_2 $ see that
\begin{align*}
    \mu_2(E \cap T_2^{-n}F) &= \mu_1( \pi^{-1}(E \cap T_2^{-n} F)) \\
        &= \mu_1( \pi^{-1}(E) \cap T_1^{-n} \pi^{-1}(F))
\end{align*}
and since $ \mu_1 $ is mixing then 
\begin{align*}
    \lim_{n\lra\infty} \mu_1( \pi^{-1}(E) \cap T_1^{-n} \pi^{-1}(F)) &= \mu_{1}(\pi^{-1}(E)) \mu_{1}(\pi^{-1}(F)) \\
        &= \mu_2(E)\mu_2(F)
\end{align*}
and so
\begin{align*}
    \lim_{n\lra\infty}\mu_2(E \cap T_2^{-n}F) = \mu_2(E)\mu_2(F)
\end{align*}

\section{}
Consider $ g_t : S^1 \lra S^1 $ where $ g_t(x) \mapsto x + t (\mod 1)$. The measure space is $ (S^{1}, \mu_{\mathcal{L}}) $ where $ \mu_{\mathcal{L}} $ is the Lebesgue measure. Clearly the flow is ergodic because the only sets which are invariant for all $ t $ are $ S $ and $ \emptyset $. However if we consider $ g_\tau $ with $ \tau = \frac{1}{2} $ then there are lots of invariant sets which are not of full or zero measure. For example $ [0, \frac{1}{4}] \cup [\frac{1}{2}, \frac{3}{4}] $ is invariant but $ \mu_{\mathcal{L}}(  [0, \frac{1}{4}] \cup [\frac{1}{2}, \frac{3}{4}]) = \frac{1}{2} $ and so $ g_\frac{1}{2} $ with the Lebesgue measure is not ergodic. 

\section{}
(We show the contrapositive)

Suppose $ (X, \mathcal{B}, \mu, T) $ is a p.p.t which is not ergodic. Then there exists another $ T $ invaraint measure $ \nu $. Since $ (X, \mathcal{B}, \mu, T) $ is not ergodic then there exists $ E \in \mathcal{B} $ such that
$ T^{-1} E = E $ but $ 0 < \mu(E) < 1 $. Fix $ E $ and say $ \mu(E) = \alpha $. Then we can define a new measure
\begin{align*}
    \nu(F) = (1-\alpha)^{-1} \mu(F \setminus E).
\end{align*}
$ \nu $ is clearly $ \mathcal{B} $ additive so it's a measure. We just have to verify that $ \nu $ is $ T $ invariant. 
\begin{align*}
    \nu(T^{-1} F) &= (1-\alpha)^{-1} \mu((T^{-1} F) \setminus E) \\
        &= (1-\alpha)^{-1} \mu(\{ x : T(x) \in F, x \in E^c \}) \\
        &= (1-\alpha)^{-1} \mu(\{ x : T(x) \in (F \cap E^c) \}) & \text{ because }  E^c \text{ is } T \text{ invariant } \\
        &= (1-\alpha)^{-1} \mu(T^{-1} (F \setminus E)) \\
        &= (1-\alpha)^{-1} \mu(F \setminus E) \\
        &= \nu(F)
\end{align*}
and so the contrapositive is proved.

\section{}
\begin{align*}
    A = \left[ 
        \begin{array}{cccccc}
            0 & 1 & 0 & 0 & 0 & 0 \\
            0 & 0 & 1 & 0 & 0 & 0 \\
            0 & 0 & 0 & 1 & 0 & 1 \\
            1 & 0 & 0 & 0 & 0 & 0 \\
            0 & 0 & 0 & 1 & 0 & 1 \\
            0 & 0 & 0 & 0 & 1 & 0 
        \end{array}
    \right]
\end{align*}
$ A $ is aperiodic and because $ A^n_{i,j} $ represents the number of paths of length $ n $ from $ i $ to $ j $ then
if $ A^{n}_{1,1} \neq 0 $ then $ A^n_{1,2} = 0 $. In other words there will never be a $ n $ such that all the entries of the matrix $ A^n $ are non-zero. 

Another way of saying this is that state $ 2 $ can only be reached from state $ 1 $ in an odd number of steps whereas state $ 1 $ can only get back to state $ 1 $ in an even number of steps.

\section{}
\subsection{}
(By induction)

In the case $ n = 1 $ the result holds trivially. 

Suppose $ A^n_{i,j} = \#\{ \text{paths from } i \text{ to } j \text{ of length } n \} $ (For all $ j, k \leq N $.)
Then 
\begin{align*}
    A^{n+1}_{i,j} &= (A^nA)_{i,j} \\
    &= \sum_{k=1}^N A^n_{i,k} A_{k,j} \\
    &= \sum_{k=1}^N \#\{ \text{paths from } i \text{ to } k \text{ of length } n \} A_{k,j} \\ 
    &= \sum_{k \in \{ 1, \ldots, N \}, A_{k,j} = 1 }\#\{ \text{paths from } i \text{ to } k \text{ of length } n \}  \\
    &= \#\{ \text{paths from } i \text{ to } j \text{ of length } n \} 
\end{align*}

\subsection{}
(By induction)

The case $ n = 1 $ holds by definition of $ P $.

Suppose $ P^n_{i,j} = \mathcal{P}[ \text{ going from } i \text{ to } j \text{ in exactly } n \text{ steps }]. $

Then
\begin{align*}
    P_{i,j}^{n+1} &= (P^nP)_{i,j} \\
        &= \sum_{k = 1}^N P^n_{i,k} P_{k,j} \\
        &= \sum_{k=1}^N P_{k,j} \mathcal{P}[ \text{ going from } i \text{ to } k \text{ in exactly } n \text{ steps }] \\
        &= \sum_{k=1}^N \mathcal{P}[ \text{ going from } k \text{ to } j \text{ in exactly } 1 \text{ step }] \mathcal{P}[ \text{ going from } i \text{ to } k \text{ in exactly } n \text{ steps }] \\
        &= \sum_{k=1}^N \mathcal{P}[ \text{ going from } i \text{ to } k \text{ in exactly } n \text{ steps then directly to } j] \\
        &= \mathcal{P}[ \text{ going from } i \text{ to } j \text{ in exactly } n+1 \text{ steps }].
\end{align*}

\subsection{}
\subsubsection{}
    \begin{align*}
        q_j &= \sum_{k=1}^{N} P_{k,j}q_j \\
            &= P_{j,j} q_j
    \end{align*}
    which implies either $ P_{j,j} = 1 $ or $ q_j = 0 $. Suppose $ P_{j,j} = 1 $, then $ P_{j,k} = 0 $ for all $ k \neq j $. This implies
    \begin{align*}
        P_{j,k}^2 &= \sum_{i = 1}^N P_{j,i}P_{i,k} \\
            &= P_{j,j}P_{j,k} \\
            &= \delta_{j,k}
    \end{align*}
    and by induction implies 
     $ (P^n)_{j,k} = 0 $ for all $ j \neq k $, which is not allowed, so $ q_j = 0 $.
\subsubsection{}
  Group into communicating classes $ [i] $, $ [j] $ respectively. (There may be other communicating classes but we can neglect them) Then there is a stochastic matrix
  $$
    R = \left[ 
    \begin{array}{cc}
      \alpha & \alpha-1 \\
      0 & 1 \\
    \end{array}
    \right]
  $$
  If $ \mathbf{w} $ is the stationary vector for $ R $ (defined above) then it is easy to see $ w_1 = 0 $.
  The result for $ q_i $ follows as $ i \in [i] $. 

\end{document}
